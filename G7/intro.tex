\chapter{Introduction}

\begin{definition}{Produit dérivé}{}
	Un produit dérivé est un outil financier dont la valeur dépend sur (ou est dérivé par) la valeur d'un autre variable sous-jancente et plus simple. 
\end{definition}

Les produits sont échangés sur différentes plateformes : 

\begin{itemize}
	\item \textbf{Marché de produits dérivés} Un marché de produits dérivés est un marché où les individus échangent des contrats standardisés, défini par le marcher d'échanges.  
	\item \textbf{Marchés en vente libre} Lorsqu'un entente est fait entre deux parties (gré à gré), il est dit vendu en vente libre (over-the-counter). Les parties présentent le contrat à une contrepartie centrale pour réduire le risque de défaut. 
\end{itemize}

Une opération au comptant (spot contract) est une opération qui implique la vente (presque) immédiate d'un titre. La valeur de l'opération au comptant est le cours au comptant (spot price). 

\section{Contrats à termes}

Les contrats à termes sont des ententes entre deux parties qui oblige la première partie à acheter à la seconde partie un actif sous-jacent à une date et un prix pré-déterminé. 

\begin{itemize} 
	\item Un contrat à terme de gré à gré (forward contract) est négocié entre deux parties. Il est échangé sur un marché en vente libre. Ces contrats ont l'avantage d'être flexibles. 
	\item Un contrat à terme standardisé (futures contract) est négocié sur un marché organisé. Le marché offre un mécanisme qui garantie que les contrats seront honorés. 
\end{itemize}

La partie qui accepte d'acheter le sous-jacent prend une position longue. La partie qui accepte de vendre le sous-jacent prend une position courte. En général, le détenteur de la position longue bénéficie d'une augmentation du prix d'un sous-jacent. 

Les contrats à terme gré à gré peuvent être utilisés pour couvrir (hedge) le risque de devise étrangère. 

Le paiement à terme de la position longue d'un \textit{forward} pour une unité du sous-jacent est $$S_T - K,$$ où $K$ est le prix d'exercice, $T$ est la date de maturité et $S_t$ est le prix du sous-jacent au temps $t$. Le paiement à terme de la position courte d'un forward est $$K - S_T.$$

\begin{tikzpicture}
\begin{axis}[
scale=0.8, 
axis lines = middle,
xmin=0, xmax = 12,
ymin=-6, ymax=6,
xlabel=$S_T$,
ylabel={Paiement à terme},
title={Position longue}, 
xtick={5}, 
xticklabels={$K$},
ytick={0}, 
]

\addplot [
domain=0:10, 
samples=100, 
color=dlblue,
]
{x-5};	

\end{axis}
\end{tikzpicture}\hfill 
\begin{tikzpicture}
\begin{axis}[
scale=0.8, 
axis lines = middle,
xmin=0, xmax = 12,
ymin=-6, ymax=6,
xlabel=$S_T$,
ylabel={Paiement à terme},
title={Position courte}, 
xtick={5}, 
xticklabels={$K$},
ytick={0}, 
]

\addplot [
domain=0:10, 
samples=100, 
color=dlblue,
]
{5-x};	

\end{axis}
\end{tikzpicture}

\section{Options}

\begin{definition}{}{}
	\begin{itemize}
		\item Un option d'achat (call option) donne le \textbf{droit} au détenteur d'acheter le sous-jacent d'ici une certaine date à un certain prix. 
		\item Un option de vente (put option) donne le \textbf{droit} au détenteur de vendre le sous-jacent d'ici une certaine date à un certain prix. 
	\end{itemize}
	Le choix d'exercer l'option ou non est toujours à la discrétion de la partie avec la position longue. C'est cette subtilité qui distingue les contrats à terme. 
	
	Le prix dans le contrat est le prix d'exercice (exercise ou strike price). La date u contrat est la date de maturité. 
	
	Une option américaine peut être exercé à tout moment jusqu'à la maturité. Une option européenne peut seulement être exercé à la maturité. 
\end{definition}


\section{Participants dans les marchés}

\begin{itemize}
	\item Gestionnaires de risque (hedgers) : utilisent des produits dérivés pour couvrir leurs risque, i.e. réduire ou limiter leur exposition au risque du marché. Ils veulent réduire leur exposition aux mouvements adverses du prix d'un sous-jacent. On peut couvrir un risque avec un contrat forward (neutraliser le risque) ou avec un contrat d'option (assurance contre une baisse ou hausse). 
	\item Spéculateurs : désirent prendre une position dans le marché et exploiter un mouvement dans le prix d'un sous-jacent. Un spéculateur qui utilise des contrats à terme est exposé à beaucoup de risque. Un spéculateur qui utilise de contrats d'options a sa perte limité par le prix de l'option. 
	\item Arbitragiste : personne qui exploite un profit sans risque en entrant simultanément dans des positions en deux ou plusieurs marchés. 
\end{itemize}

Il peut être plus bénéfique pour les gestionnaires de risques et les spéculateurs d'utiliser des produits dérivés que le produit dérivé lui-même. 

