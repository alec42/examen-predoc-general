\chapter{Caractéristiques des modèles actuariels}

Voir \cite{klugman2012loss}, 

Dans ce chapitre, les modèles sont caractérisés par combien d'information est nécessaire pour spécifier le modèle. Le nombre de paramètre donne une indication de la complexité du modèle. 

\section{Distributions paramétriques et d'échelle}

\begin{definition}{Distribution paramétrique}{}
	Ensemble de distributions déterminée par une ou plus valeurs appelé(s) paramètre. Le nombre de paramètre est fixe et connu
\end{definition}

\begin{definition}{Distribution d'échelle}{}
	Si une variable aléatoire est multipliée par une constante positive, et que la nouvelle variable aléatoire est dans le même ensemble de distirbutions, elle est dite une distribution d'échelle.
\end{definition}

\begin{definition}{Paramètre d'échelle}{}
	Un paramètre d'échelle satisfait deux conditions : 
	\begin{enumerate}
		\item Lorsqu'une distribution d'échelle est multipliée par une constante, ce paramètre d'échelle est multiplié par la même constante.
		\item Tous les autres paramètres de la distribution restent inchangés (le paramètre d'échelle \textit{absorbe} tout le changement d'échelle).
	\end{enumerate}
\end{definition}

\section{Familles de distributions paramétriques}

\begin{definition}{Familles de distributions paramétriques}{}
	Ensemble de distributions qui sont reliées de manière significative. Exemple : la loi exponentielle est un cas particulier de la loi gamma avec $\alpha = 1$. Alors, la loi exponentielle est reliée de manière significative avec la loi gamma. 
\end{definition}

\section{Distribution de mélange fini}

\begin{definition}{Mélange $k$-point}{}
	Une distribution est un mélange $k$-point si 
	$$F_Y(y) = a_1 F_{X_1}(y) + a_2 F_{X_2}(y) + \dots + a_k F_{X_k}(y)$$
	avec $a_1 + a_2 + \dots + a_k = 1.$
\end{definition}

\begin{definition}{Mélange à composante variable}{}
	Une distribution mélange $k$-point où $K$ n'est pas fixé. 
	\begin{itemize}
		\item $\displaystyle F(x) = \sum_{j = 1}^{K}a_j F_j(x)$
		\item $\displaystyle \sum_{j = 1}^{K}a_j = 1, \quad a_j > 0, j = 1, \dots, K, \quad K = 1, 2, \dots$
		\item Le nombre de paramètres est la somme des paramètres de chaque distribution plus ($K$-1) car le dernier paramètre $a_k$ peut être calculé avec $1 - a_1 - a_2 - \dots - a_{K-1}$.
	\end{itemize}
	Ce modèle est appelé semi-paramétrique car sa complexité est entre un modèle paramétrique et non-paramétrique
\end{definition}

\section{Distributions dépendantes des données}

\begin{itemize}
	\item Une distribution dépendante des données est au moins aussi compliqué que les données ou l'information produites par ces données. 
	\item Le nombre de paramètres augmente lorsque le nombre de données augmente
	\item Exemple : un modèle empirique est une distribution discrète basée sur la taille échantillonnale $n$ qui assigne une probabilité $\frac{1}{n}$ à chaque point
	\item Exemple : modèle de lissage à noyau
\end{itemize}







