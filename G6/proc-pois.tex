\chapter{Processus de Poisson}

\section{Distribution exponentielle}
Définition :

\begin{itemize}
	\item $\displaystyle f_X(x) = \lambda e^{-\lambda x}, \quad x \geq 0$
	\item $\displaystyle F_X(x) = 1 -  e^{-\lambda x}, \quad x \geq 0$
	\item Premier moment : $\displaystyle E[X] = \frac{1}{\lambda}$
	\item FGM : $\displaystyle M_X(t) = \frac{\lambda}{\lambda - t}, \quad \text{où } t < \lambda$
	\item Fonction d'intensité : $\displaystyle r(t) = \frac{f(t)}{1 - F(t)}$
\end{itemize}

Propriétés : 

\begin{itemize}
	\item Sans mémoire : $\displaystyle \Pr(X > s + t \vert X > t) = \Pr(X > s)$
	\item Probabilité avec les minimums : $\displaystyle \Pr(X_1 < X_2) = \frac{\lambda_1}{\lambda_1 + \lambda_2}$
	\item Probabilité avec les minimums ($n$ v.a.): $\displaystyle \Pr(X_i  = \min(X_1, \dots, X_n)) = \frac{\lambda_i}{\sum\lambda_j}$
	\item Distribution des minimums : exponentielle avec somme des intensités
\end{itemize}

MANQUE ERLANG GÉNÉRALISÉE

\section{Processus de Poisson}

\begin{definition}{Processus de comptage}{}
	Un processus de comptage 
	\begin{enumerate}
		\item $N(t) \geq 0$
		\item $N(t)$ est un entier
		\item si $s<t$, alors $N(s) \leq N(t)$
		\item Pour $s<t, N(t) - N(s)$ correspond au nombre d'évènements dans l'intervalle $(s, t]$.
	\end{enumerate}
\end{definition}

Un processus de comptage a
\begin{itemize}
	\item des accroissements indépendants si le nombre d'évènements dans des intervalles distinctes sont indépendants
	\item des accroissements stationnaires si la distribution du nombre d'évènements dans l'intervalle dépend seulement de la longueur de l'intervalle. 
\end{itemize}

\begin{definition}{Processus de Poisson I}{}
	Le processus de comptage $\{N(t), t \geq 0\}$ est un processus de Poisson avec taux $\lambda > 0$ si
	\begin{enumerate}
		\item $N(0) = 0$
		\item Le processus a des accroissements indépendants
		\item Le nombre d'évènements dans l'intervalle de longueur $t$ est Poisson avec moyenne $\lambda t$
	\end{enumerate}
\end{definition}

\begin{definition}{}{}
	Une fonction $f$ est dite $o(h)$ si 
	$$\lim\limits_{h\to 0} \frac{f(h)}{h} = 0.$$
\end{definition}

\begin{definition}{Processus de Poisson II}{}
	Le processus de comptage $\{N(t), t\geq 0\}$ est dit un processus de Poisson avec taux d'intensité $\lambda > 0$ si
	\begin{enumerate}
		\item $N(0) = 0$
		\item Le processus a des accroissements indépendants et stationnaires
		\item $\Pr(N(h) = 1) = \lambda h + o(h)$
		\item $\Pr(N(h) = 2) = o(h)$
	\end{enumerate}
\end{definition}

\begin{definition}{}{}
	La séquence de temps entre les arrivés est notée par $\{T_n, n = 1, 2, \dots\}$. Le temps du premier sinistre est $T_1$. Avec 
	$$\Pr(T_1 > t) = \Pr(N(t) = 0) = e^{-\lambda t},$$
	donc le temps du premier sinistre est de distribution exponentielle avec paramètre $\lambda$. Le temps entre le premier et le deuxième sinistre $T_2$ est 
	$$\Pr(T_2 > t \vert T_1 = s) = \Pr(N(t + s) - N(s) = 0) = e^{-\lambda (t + s - s)} = e^{-\lambda t}.$$
	\tcblower
	$T_n, n = 1, 2, \dots$ sont iid et de distribution exponentielle avec paramètre $\lambda$. 
\end{definition}

\begin{definition}{}{}
	Le temps d'attente au $n^\text{e}$ évènement est noté par la v.a. $S_n = \sum_{i = 1}^{n}T_i, n\geq 1$. De plus, on a $S_n \sim Gamma(n, \lambda)$.
\end{definition}

Les énoncés suivants sont équivalents : 
$$N(t) \geq n \Leftrightarrow S_n \leq t.$$

\begin{propriete}{}{}
	Soient $N_1(t)$ et $N_2(t)$, deux processus de Poisson avec intensités respectives $\lambda p$ et $\lambda(1-p)$. Alors, $N(t) = N_1(t) + N_2(t)$ est un processus de Poisson avec intensité $\lambda$. 
\end{propriete}

La probabilité que $n$ évènements de type I (noté $S_n^1$) arrivent avant $m$ évènements de type II (noté $S_m^2$). On a $P(S_1^1 < S_1^2) = \frac{\lambda_1}{\lambda_1 + \lambda_2}$. Répéter cette expérience donne une v.a. binomiale et on a 
$$\Pr\left(S_n^1<S_m^2\right) = \sum_{k = n}^{n + m - 1} \binom{m + n - 1}{k} \left(\frac{\lambda_1}{\lambda_1 + \lambda_2}\right)^k \left(\frac{\lambda_2}{\lambda_1 + \lambda_2}\right)^{n+m-1-k}.$$

\begin{proposition}{}{}
	Si 1 évènement d'un processus de Poisson a eu lieu au temps $t$, le temps de cet évènement est uniforme sur $[0, t]$. En effet, on a 
	\begin{align*}
	\Pr(T_1 < s \vert N(t) = 1) &= \frac{\Pr(T_1 < s, N(t) = 1)}{\Pr(N(t) = 1)}\\
	&= \frac{\Pr(\text{1 évènement sur }[0, s), \text{0 évènement sur }[s, t])}{\Pr(N(t) = 1)}\\
	&= \frac{\Pr(\text{1 évènement sur }[0, s))\Pr(\text{0 évènement sur }[s, t])}{\Pr(N(t) = 1)}\\
	&= \frac{\lambda s e^{-\lambda s} e^{-\lambda(t-s)}}{\lambda t e^{-\lambda t}}\\
	&= \frac{s}{t}.
	\end{align*}
\end{proposition}

Une version plus générale de ce résultat est présenté dans le théorème suivant : 

\begin{theoreme}{}{}
	Sachant $N(t) = n$, les temps d'inter-arrivé $S_1, \dots, S_n$ sont distribués comme les statistiques d'ordres des $n$ v.a. uniformes sur $(0, t)$.
	\tcblower
	La densité conditionnelle de $S_1, \dots, S_n$ sachant $N(t) = n$ est équivalante à la densité conditionnelle de $T_1 = s_1, T_2 = s_2 - s_1, \dots, T_n = s_n - s_{n-1}, T_{n+1} > t - s_n$. Alors, on a 
	\begin{align*}
	f(s_1, \dots, s_n \vert n) &= \frac{f(s_1, \dots, s_n, n)}{\Pr(N(t) = n)}\\
	&= \frac{\lambda e^{-\lambda s_1} \lambda e^{-\lambda(s_2 - s_1)} \dots \lambda e^{-\lambda (s_n - s_{n-1})} e^{\lambda(t-s_n)}}{\frac{e^{-\lambda t}(\lambda t)^n}{n!}}\\
	&= \frac{n!}{t^n}, \quad 0<s_1<\dots <s_n<t.
	\end{align*}
\end{theoreme}

\begin{proposition}{}{}
	On suppose que si un évènement arrive au temps $y$, alors il sera classifié comme type $i$, indépendanment, avec probabilité $P_i(y), i = 1, \dots, k$, où $\sum_{i = 1}^{k}\pi_i(y) = 1.$ Alors, si $N_i(t), i = 1, \dots, k$ représente le nombre d'évènements de type $i$ qui arrivent d'ici le temps $t$ alors $N_i(t), i = 1, \dots, k$ sont des v.a. indépendantes avec moyennes 
	$$E[N_i(t)] = \lambda \int_{0}^{t}P_i(s) ds. $$
\end{proposition}

\section{Généralisations du processus de Poisson}

\begin{definition}{Processus de Poisson non homogène}{}
	Le processus de comptage $\{N(t), t\geq 0\}$ est un processus de Poisson non-homogène avec fonction d'intensité $\lambda(t), t\geq 0$ si 
	\begin{enumerate}
		\item $\displaystyle N(0) = 0$
		\item $\displaystyle \{N(t), t\geq t\}$ a des accroissements indépendants
		\item $\displaystyle \Pr(N(t + h) - N(t) \geq 2) = o(h)$
		\item $\displaystyle \Pr(N(t + h) - N(t) = 1) = \lambda(t) h + o(h)$
	\end{enumerate}
\end{definition}

\begin{proposition}{}{}
	Soient $\{N(t), t \geq 0\}$ et $\{M(t), t\geq 0\}$, des processus de Poisson non-homogènes indépendants, avec fonctions d'intensités respectives $\lambda(t)$ et $\mu(t)$, et soit $N^*(t) = N(t) + M(t)$. Alors, 
	\begin{enumerate}
		\item $\{N^*(t), t\geq 0\}$ est un processus de Poisson non homogène avec fonction d'intensité $\lambda(t) + \mu(t)$
		\item Sachant qu'un évènement de type $t$ arrive au temps $t$, alors, indépendament de ce qu'il se passe avant $t$, l'évènement au temps $t$ provient du processus $\{N(t)\}$ avec probabilité $\frac{\lambda(t)}{\lambda(t) + \mu(t)}$
	\end{enumerate}
\end{proposition}

% RENDU 341

Un avantage du processus de Poisson non-homogène est que l'hypothèse d'accroissements non stationnaires n'est plus requise. De plus, le nombre d'évènements entre $s$ et $s + t$ est Poisson avec moyenne $\int_{s}^{s+ t}\lambda(y) dy.$

\begin{definition}{}{}
	Le temps du $n^\text{e}$ évènement d'un processus de Poisson non-homogène est 
	$$f_{S_n}(t)= \lambda(t) e^{-m(t)}\frac{[m(t)]^{n-1}}{(n-1)!},$$
	où $m(t) = \int_{0}^{t}\lambda(s) ds.$
\end{definition}

\subsection{Processus de Poisson composé}

Un processus stochastique $\{X(t), t\geq 0\}$ est un processus de Poisson composé s'il peut être représenté par le modèle stochastique 
$$X(t)= \sum_{i = 1}^{N(t)}Y_i, \quad t\geq 0,$$
où $\{N(t), t \geq 0\}$ est un processus de Poisson et $Y_i$ sont des v.a. iid et indépendantes du processus de comptage. On a 
$$E[X(t)] = \lambda t E[Y] \quad \text{ et } \quad Var(X(t)) = \lambda t E[Y^2].$$

\begin{propriete}{}{}
	Soit $\{X(t), t \geq 0\}$, un processus de Poisson composé avec intensité $\lambda_1$ et fonction de répartition $F_1$. Soit aussi $\{Y(t), t \geq 0\}$, un processus de Poisson composé avec $\lambda_2$ et $F_2$. Alors, $\{X(t) + Y(t),t\geq 0\}$ est un processus de Poisson composé avec intensité $\lambda_1 + \lambda_2$ et fonction de répartition
	$$F(x) = \frac{\lambda_1}{\lambda_1 + \lambda_2} F_1(x) + \frac{\lambda_2}{\lambda_1 + \lambda_2} F_2(x).$$
\end{propriete}

\subsection{Processus de Poisson conditionnel ou mixte}

Soit $\{N(t), t\geq 0\}$, un processus de Poisson avec paramètre $\lambda$. Soit $\Lambda$, une v.a. positive représentant l'hétérogénéité du paramètre $\lambda$. Ce processus de Poisson est appelé conditionnel ou mixte. Stratégie générale : conditionner. On a 
\begin{align*}
	\Pr(N(t + s) - N(s) = n) &= \int_{0}^{\infty}\Pr(N(t + s) - N(s) = n \vert \Lambda=\lambda) f_\Lambda(\lambda)d\lambda\\
	&= \int_{0}^{\infty}\frac{(\lambda t)^n e^{-\lambda t}}{n!} f_\Lambda(\lambda)d\lambda\\
\end{align*}
On a aussi 
$$Var(N(t)) = tE[L] + t^2Var(L).$$
De plus, 
$$\Pr(\Lambda \leq x \vert N(t) = n) = \frac{\int_{0}^{x}(\lambda t)^ne^{-\lambda t}f_\Lambda(\lambda) d\lambda }{\int_{0}^{\infty}(\lambda t)^ne^{-\lambda t}f_\Lambda(\lambda) d\lambda }$$
et
$$f_{\Lambda \vert N(t) = n}(\lambda \vert n) = \frac{(\lambda t)^ne^{-\lambda t}f_\Lambda(\lambda)}{\int_{0}^{\infty}(\lambda t)^ne^{-\lambda t}f_\Lambda(\lambda) d\lambda }.$$
Enfin, 
$$\Pr(N(t) > n) = \int_{0}^{\infty}\overline{F}_\Lambda(\lambda)\lambda \frac{(\lambda t)^n e^{-\lambda t}}{n!}d\lambda.$$












































































































