\chapter{Comparaison des risques}

L'ordre en dominance stochastique permet de comparer l'ampleur de deux v.a. et l'ordre convexe compare leur variabilité. 

\section{Ordres partiels}

\begin{definition}{}{}
	Soit $A$ un ensemble de fonctions de répartition. Une relation binaire $\preceq$ définie sur $A$ est un ordre partiel si cet ordre satisfait les trois propriétés suivantes : \begin{enumerate}
		\item Transitivité : Si $F_X \preceq F_Y$ et $F_Y \preceq F_Z$, alors $F_X \preceq F_Z$. 
		\item Réflexicité : $F_X \preceq F_Y$
		\item Anti-symétrie : Si $F_X \preceq F_Y$ et $F_Y \preceq F_X$, alors $F_X \equiv F_y$.
	\end{enumerate}
\end{definition}

Les propriétés désirables pour les ordres partiels sont les suivants : 

\begin{propriete}{Fermeture sous le mélange}{}
	Soient les v.a. $X, X'$ et $\Theta$. On dit que l'ordre stochastique $\preceq$ est fermé sous le mélange si $X\vert \Theta = \theta \preceq X'\vert \Theta = \theta$ pour tout $\theta$ implique $X \preceq X'$. 
\end{propriete}

\begin{propriete}{Fermeture sous la convolution}{}
	Soient deux suites de v.a. indépendantes $\underline{X} = \left\{X_i, i \in \mathbb{N}^+ \right\}$ et $\underline{X'} = \left\{X_i', i \in \mathbb{N}^+ \right\}$ avec $X_i \preceq X_i'$ pour tout $i \in \mathbb{N}^+$. On dit que l'ordre stochastique $\preceq$ est fermé sous la convolution si $$\sum_{i = 1}^{n}X_i \preceq \sum_{i = 1}^{n}X_i'$$ est satisfait.
\end{propriete}

\begin{propriete}{Fermeture sous la composition}{}
	On dit que la 'ordre est fermé sous la composition si $$\sum_{i = 1}^{N}X_i \preceq \sum_{i = 1}^{N'}X_i$$ pour $N \preceq N'$. 
\end{propriete}

\section{Ordre en dominance stochastique}

\begin{definition}{}{}
	Soient les v.a. $X$ et $X'$ telles que $E[X] < \infty$ et $E[X']< \infty$. Alors, $X$ est inférieure à $X'$ sous l'ordre en dominance stochastique, notée $X \preceq_{sd}X'$, si $F_X(x) \geq F_{X'}(x)$ ou $\overline{F}_X(x) \leq \overline{F}_{X'}(x)$ pour tout $x\in \mathbb{R}$. 
\end{definition}

\begin{proposition}{}{}
	Soient deux v.a. $X$ et $X'$ telles que $X \preceq_{sd} X'$. Alors, on a $VaR_\kappa(X) \leq VaR_\kappa(X')$. 
\end{proposition}

\begin{theoreme}{}{}
	Soient les v.a. $X$ et $X'$ telles que $E[X] <\infty$ et $E[X'] < \infty$. 
	\begin{enumerate}
		\item Si $X \preceq_{sd} X'$ alors $E[X] \leq E[X']$. 
		\item Si $X \preceq_{sd} X'$ et $E[X] = E[X']$ alors $X$ et $X'$ ont la même distribution. 
	\end{enumerate}
\end{theoreme}

\begin{theoreme}{}{}
	Soient les deux v.a. $X$ et $X'$. Alors, on a $X\preceq_{sd} X'$ si et seulement si $E[\phi(X)] \leq E[\phi(X')]$ pour toute fonction croissante $\phi$. 
\end{theoreme}

\begin{theoreme}{}{}
	Soient les v.a. $X$ et $X'$ telles que $E[X^m] < \infty$ et $E[X'^m]<\infty$, pour $m\in\mathbb{N}^+$. Si on a $X \preceq_{sd} X'$, alors
	\begin{enumerate}
		\item $E[X^m]\leq E[X'^m]$, pour $m = 1, 3, 5, \dots$ ;
		\item $E[X^m]\leq E[X'^m]$, pour $m = 1, 2, 3,  \dots$ si les v.a. $X$ et $Y$ sont positives. 
	\end{enumerate}
\end{theoreme}

\begin{theoreme}{}{}
	Soient deux v.a. $X$ et $X'$. Si $X \preceq_{sd} X'$ alors on a $\phi(X) \preceq_{sd} \phi(X')$ pour toute fonction croissante $\phi$. 
	\tcblower
	Soient $X_1, \dots, X_n$ et $X_1', \dots, X_n'$ des v.a. indépendantes avec $X_i \preceq_{sd} X_i'$ pour $i = 1, 2, \dots, n$. On suppose que $\psi : \mathbb{R}^+ \to \mathbb{R}$ est une fonction croissante. Alors, on a $\psi(X_1, \dots, X_n) \preceq_{sd} \psi(X_1', \dots, X_n')$.  
\end{theoreme}

L'ordre en dominance stochastique est fermé sous la convolution, le mélange et la composition.

\begin{proposition}{Condition suffisante pour déterminer l'ordre en dominance stochastique}{}
	Soient deux v.a. $X$ et $Y$ ayant la propriété qu'il existe un nombre réel $a\geq 0$ tel que 
	\begin{align*}
	dF_X(x) &\geq dF_Y(x) \text{ pour } x\in(-\infty, a)\\
	dF_X(x) &\leq dF_Y(x) \text{ pour } x\in(a, \infty).
	\end{align*}
	Alors, $X\preceq_{sd} Y$. 
\end{proposition}

Quelques lois avec paramètres différents sont d'ordre en dominance stochastique. Par exemple, 
$$X \sim Exp(\beta) \text{ et } X' \sim Exp(\beta') \text{ avec } \beta \geq \beta'.$$
Il est aussi possible de comparer des lois paramétriques différentes. Par exemple, si $M \sim Bern(q)$ et $M' \sim Pois(\lambda)$ où $\lambda = -\ln (1 - q)$. Alors, on a $M \preceq_{sd} M'$.

\begin{proposition}{}{}
	Soient les v.a. $\widetilde{X}^{(u, h')}, \widetilde{X}^{(u, h)}, X, \widetilde{X}^{(l, h)}, \widetilde{X}^{(l, h')}$, où $X$ est un modèle stochastique $\widetilde{X}$ est l'approximation upper ou lower pour $B$. Alors, 
	$$\widetilde{X}^{(u, h')} \preceq_{sd}\widetilde{X}^{(u, h)}\preceq_{sd} X\preceq_{sd} \widetilde{X}^{(l, h)}\preceq_{sd} \widetilde{X}^{(l, h')}$$
	pour $0<h\leq h$. 
\end{proposition}

\begin{definition}{}{}
	Une v.a. $X_\theta$ est dite stochastiquement croissante si et seulement si $X_\theta \preceq_{sd} X_{\theta'}$ quand $\theta\leq \theta'$. 
\end{definition}

\begin{proposition}{}{}
	Si $\Theta \preceq_{sd} \Theta'$, alors $X_\Theta \preceq_{sd} X_{\Theta'}$. 
\end{proposition}

\section{Ordres convexes}

\begin{definition}{}{}
	Une fonction est dite convexe si $\phi(\alpha x + (1-\alpha)y) \leq \alpha \phi(x) + (1-\alpha) \phi(y)$ pour tout $x$ et $y$ et pour tout $0<\alpha<1$. 
	\tcblower
	Une fonction $\phi$ est dite concave si $-\phi$ est convexe. 
\end{definition}

On définit des ordres stochastiques. 

\begin{definition}{}{}
	Soient $X$ et $X'$ telles que $E[X] < \infty$ et $E[X'] < \infty$. Alors, 
	\begin{itemize}
		\item Ordre convexe : $X\preceq_{cx} X'$ si $E[\phi(X)] \leq E[\phi(X')]$, pour toute fonction réelle \textit{convexe} $\phi$ et en supposant que les espérances existent. 
		\item Ordre convexe croissante : $X\preceq_{icx} X'$ si $E[\phi(X)] \leq E[\phi(X')]$, pour toute fonction réelle \textit{convexe} \textit{croissante} $\phi$ et en supposant que les espérances existent. 
		\item Ordre concave croissante : $X\preceq_{icv} X'$ si $E[\phi(X)] \leq E[\phi(X')]$, pour toute fonction réelle \textit{concave} \textit{croissante} $\phi$ et en supposant que les espérances existent. 
	\end{itemize}
\end{definition}

\begin{definition}{Ordre stop-loss}{}
	$X\preceq_{sl}X'$, si $E[(X-d)_+] \leq E[(X'-d)_+]$, pour tout $d\in \mathbb{R}$, où $(u)_+ = \max(u;0)$.
\end{definition}

\begin{theoreme}{}{}
	Les propositions suivantes sont équivalentes : 
	\begin{enumerate}
		\item $X \preceq_{icx} X'$ ;
		\item $X \preceq_{sl}$. 
	\end{enumerate}
\end{theoreme}

\begin{remarque}{}{}
	$X\preceq_{sd} X'$ implique $X \preceq_{icx} X'$ mais pas $X\preceq_{cx} X'$. 
\end{remarque}

Les ordres convexes, convexes croissantes et \textit{stop loss} sont fermés sous la convolution, le mélange et la composition.

\begin{proposition}{Critère de Karlin-Novikoff}{}
	Soient deux v.a. $X$ et $X'$ telles que $E[X] \leq E[X'] < \infty$. S'il existe un nombre $c>0$ tel que 
	\begin{align*}
	F_X(x) \leq F_{X'}(x), \quad x<c\\
	F_X(x) \geq F_{X'}(x), \quad x\geq c,
	\end{align*}
	alors $X\preceq_{icx} X'$ ou $X\preceq_{sl} X'$. 
\end{proposition}

\begin{proposition}{}{}
	Soient les v.a. $X$ et $X'$. Alors, on a $X\preceq_{icx} X'$ si et seulement si $TVaR_\kappa(X) \leq TVaR_\kappa(X')$, pour tout $0<\kappa<1$. 
\end{proposition}

\begin{proposition}{}{}
	Soient la v.a. positive $\Theta$ et la v.a. $X_\theta \sim Pois(\lambda \theta)$, pour $\theta \in \mathbb{R}^+$. Les v.a. $\Theta$ et $X_\theta$ sont indépendantes. Alors, $\Theta \preceq_{icx} \Theta'$ implique $X_\Theta \preceq_{icx} X_{\Theta'}$. 
\end{proposition}

\begin{proposition}{}{}
	On a $W_{n+1} \preceq_{icx} W_n$ pour tout $n\in \mathbb{N}^+$.
\end{proposition}



























