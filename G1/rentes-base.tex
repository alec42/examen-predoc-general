\chapter{Rentes de base}
Voir \cite{kellison2006theory}, chapitre 3
\section{Introduction}

\begin{definition}{Rente}{}
	\begin{itemize}
		\item Une rente est une série de paiements fait à intervalles égaux. 
		\item Une rente certaine est une rente dont les paiements sont faits pour une période de temps avec certitude. 
		\item L'intervalle entre les paiements est la période de paiements
	\end{itemize}
\end{definition}

\section{Rente immédiate}

Une rente immédiate paie 1 à la fin de chaque période pour $n$ périodes. 

\begin{center}
	\begin{tikzpicture}
	\draw (0,0) -- (8,0);
	\draw (0,0.1) -- + (0,-0.2) node[below] {$0$};
	\foreach \i in {1,...,3} {
		\draw (\i,0.1) -- + (0,-0.2) node[below] {$\i$};
		\draw (\i,0.1) -- + (0,-0.2) node[above, yshift=8pt] {$1$};}
	\draw (8,0.1) -- + (0,-0.2) node[below] {$n$};
	\draw (8,0.1) -- + (0,-0.2) node[above, yshift=8pt] {$1$};
	\draw (5,0) node[above, yshift = 8pt] {$\dots$};
	\draw (5,0) node[below, yshift = -8pt] {$\dots$};
	\draw (7,0.1) -- + (0,-0.2) node[below] {$n-1$};
	\draw (7,0.1) -- + (0,-0.2) node[above, yshift=8pt] {$1$};
	\draw (0,0) node[below, yshift = -15pt, scale = 1.5, color = dlblue] {$t_1$};
	\draw (8,0) node[below, yshift = -15pt, scale = 1.5, color = dlblue] {$t_2$};
	\end{tikzpicture}
\end{center}

La valeur actualisée d'une rente immédiate (au temps \textcolor{dlblue}{$t_1$}) est notée par $\ax{\angl{n}}$. On a 
$$\ax{\angl{n}} = v + v^2 + v^3 + \dots + v^{n-1} + v^n = \frac{1 - v^n}{i}.$$

La valeur accumulée d'une rente immédiate (au temps \textcolor{dlblue}{$t_2$}) est notée par $\sx{\angl{n}}$
$$\sx{\angl{n}} = 1 + (1+i) + (1+i)^2+ \dots + (i+1)^{n-1} + (1+i)^n = \frac{(1+i)^n - 1}{i}.$$

Quelques relations 
\begin{multicols}{3}
	\begin{itemize}
		\item $1 = i\ax{\angl{n}} + v^n$
		\item $\sx{\angl{n}} = \ax{\angl{n}} (1+i)^n$
		\item $\displaystyle \frac{1}{\ax{\angl{n}}} = \frac{1}{\sx{\angl{n}}} + i$
	\end{itemize}
\end{multicols}

\section{Rente due}

Une rente due paie 1 au début de chaque période pour $n$ périodes. 

\begin{center}
	\begin{tikzpicture}
	\draw (0,0) -- (8,0);
	\foreach \i in {0,...,2} {
		\draw (\i,0.1) -- + (0,-0.2) node[below] {$\i$};
		\draw (\i,0.1) -- + (0,-0.2) node[above, yshift=8pt] {$1$};}
	\draw (8,0.1) -- + (0,-0.2) node[below] {$n$};
%	\draw (8,0.1) -- + (0,-0.2) node[above, yshift=8pt] {$1$};
	\draw (4,0) node[above, yshift = 8pt] {$\dots$};
	\draw (4,0) node[below, yshift = -8pt] {$\dots$};
	\draw (7,0.1) -- + (0,-0.2) node[below] {$n-1$};
	\draw (7,0.1) -- + (0,-0.2) node[above, yshift=8pt] {$1$};
	\draw (6,0.1) -- + (0,-0.2) node[below] {$n-2$};
	\draw (6,0.1) -- + (0,-0.2) node[above, yshift=8pt] {$1$};
	\draw (0,0) node[below, yshift = -15pt, scale = 1.5, color = dlblue] {$t_1$};
	\draw (8,0) node[below, yshift = -15pt, scale = 1.5, color = dlblue] {$t_2$};
	\end{tikzpicture}
\end{center}

La valeur actualisée d'une rente due (au temps \textcolor{dlblue}{$t_1$}) est notée par $\ax**{\angln}$. On a 
$$\ax**{\angln} = 1 + v + v^2 \dots + v^{n-2} + v^{n-1} = \frac{1 - v^n}{d}.$$

La valeur accumulée d'une rente due (au temps \textcolor{dlblue}{$t_2$}) est notée par $\sx**{\angln}$
$$\sx**{\angln} = (1+i) + (1+i)^2 + (1+i)^3 + \dots + (i+1)^{n-2} + (i+1)^{n-1} = \frac{(1+i)^n - 1}{d}.$$

Quelques relations 
\begin{multicols}{3}
	\begin{itemize}
		\item $\sx**{\angln} = \ax**{\angln}(1+i)^n$
		\item $\displaystyle \frac{1}{\ax**{\angln}} = \frac{1}{\sx**{\angln}} + d$
		\item $\ax**{\angln} = \ax{\angln}(1+i)$
		\item $\sx**{\angln} = \sx{\angln}(1+i)$
		\item $\ax**{\angln} = 1 + \ax{\angl{n-1}}$
		\item $\sx**{\angln} = \sx{\angl{n+1}} - 1$
	\end{itemize}
\end{multicols}
Comparaison des rentes : 
\begin{center}
	\begin{tikzpicture}
	\draw (0,0) -- (8,0);
	\draw (0,0.1) -- + (0,-0.2) node[below] {$0$};
	\foreach \i in {1,...,2} {
		\draw (\i,0.1) -- + (0,-0.2) node[below] {$\i$};
		\draw (\i,0.1) -- + (0,-0.2) node[above, yshift=8pt] {$1$};}
	\draw (8,0.1) -- + (0,-0.2) node[below] {$n+1$};
	%	\draw (8,0.1) -- + (0,-0.2) node[above, yshift=8pt] {$1$};
	\draw (4,0) node[above, yshift = 8pt] {$\dots$};
	\draw (4,0) node[below, yshift = -8pt] {$\dots$};
	\draw (7,0.1) -- + (0,-0.2) node[below] {$n$};
	\draw (7,0.1) -- + (0,-0.2) node[above, yshift=8pt] {$1$};
	\draw (6,0.1) -- + (0,-0.2) node[below] {$n-1$};
	\draw (6,0.1) -- + (0,-0.2) node[above, yshift=8pt] {$1$};
	\draw (0,0) node[below, yshift = -19pt, scale = 1.5, color = dlred] {$\ax{\angln}$};
	\draw (1,0) node[below, yshift = -15pt, scale = 1.5, color = dlred] {$\ax**{\angln}$};
	
	\draw (7,0) node[below, yshift = -19pt, scale = 1.5, color = dlred] {$\sx{\angln}$};
	\draw (8,0) node[below, yshift = -15pt, scale = 1.5, color = dlred] {$\sx**{\angln}$};
	\end{tikzpicture}
\end{center}









