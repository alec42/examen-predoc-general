\chapter{Primes}

Ce chapitre s'intéresse aux primes d'assurance ou de rentes. 

\section{Principes de primes}

Les primes sont souvent basés sur le principe d'équivalance. 

\begin{definition}{}{}
	Le principe d'équivalance requiert que l'espérance de la perte $L$ soit nulle, c'est-à-dire
	$$E[L] = 0.$$
	Les primes qui satisfaient le principe d'équivalance sont appelés des primes nettes. 
	
	On a 
	$$E[\text{valeur actualisée de la prestation}] = E[\text{valeur actualisée des primes}]$$
\end{definition}

Un autre principe de prime est la prime percentile, où la prime est déterminée de telle sorte que la probabilité d'avoir une perte positive est plus petite qu'un seuil $\alpha$. 

\section{Primes pour des contrats continus}

La valeur actualisée de la perte de l'assureur est 




























