\chapter{Rentes}

\begin{definition}{}{}
	Une rente vie-entière paie des prestations jusqu'à la mort de l'individu. La variable aléatoire correspondant à la valeur actualisée de cette rente est 
	$$Y = \ax*{\angl{T}}.$$
	La fonction de répartition est 
	$$F_Y(y) = F_T\left(\frac{-\log(1-\delta y)}{\delta}\right), \quad 0<y< \frac{1}{\delta}$$
	et la fonction de densité est
	$$\frac{1}{1- \delta y} f_T\left(\frac{-\log(1-\delta y)}{\delta}\right), \quad 0<y< \frac{1}{\delta}.$$
	La valeur actualisée est 
	$$\ax*{x} = E[Y] = \int_{0}^{\infty} \ax*{\angl{x}} \px[t]{x}\mu_{x + t} dt = \int_{0}^{\infty} v^t \px[t]{x} dt = \int_{0}^{\infty} \Ex[t]{x}dt.$$
\end{definition}

On peut définir des relations de récursion comme
$$\ax*{x} = \ax{x:\angl{1}} + vp_x \ax*{x+1}.$$ Savoir interpréter ces relations (\textit{Une rente vie entière est une rente temporaire un an plus une rente vie entière dans un an, si l'assuré survit un an}).

De la définition de la rente continue, on avait
$$\ax*{\angl{t}} = \frac{1 - v^t}{\delta}.$$
En prenant l'espérance des deux côtés de l'équation, on obtient la relation 
$$\ax*{x} = \frac{1 - \Ax{x}}{\delta}.$$
On a aussi 
\begin{align*}
Var\left(\ax*{\angl{T}}\right) &= Var \left(\frac{1 - v^T}{\delta}\right)\\
&= \frac{Var(v^T)}{\delta^2} = \frac{\Ax*[][2]{x} - \left(\Ax*{x}\right)^2}{\delta^2}
\end{align*}

\section{Rentes discrètes}

\begin{definition}{Rente-due vie entière}{}
	On considère une rente qui paie 1 au début de chaque période chaque année où $(x)$ survit. 
	
	La variable aléatoire de la valeur actualisée est $$Y = \ax**{\angl{K + 1}}.$$
	
	La valeur actuarielle actualisée est 
	$$\ax**{x} = \sum_{k = 0}^{\infty} \ax**{\angl{K + 1}} \px[k]{x}q_{k + k} = \sum_{k = 0}^{\infty} v^k \px[k]{x}.$$
\end{definition}

La relation avec une rente discrète est 
$$\frac{1 - A_x}{d}.$$

\subsection{Rente temporaire}

\begin{definition}{Rente temporaire $n$ années}{}
	La variable aléatoire de la valeur actualisée pour une rente temporaire est 
	$$Y = \begin{cases}
	\ax**{\angl{K + 1}}, & 0\leq K < n \\
	\ax**{\angl{n}},     & K \geq n
	\end{cases}$$
	L'espérance de cette variable aléatoire est 
	$$\ax**{\endowxn} = \sum_{k = 0}^{n - 1} \ax**{\angl{k+1}} \px[k]{x}q_{x+k} + \ax**{\angln}\px[n]{x} = \sum_{k = 0}^{n-1}v^k \px[k]{x}.$$
	On a la relation 
	$$\frac{1 - \Ax{\endowxn}}{d}.$$
\end{definition}

\begin{definition}{Rente différée $n$ années}{}
	Une rente différée $n$ années paie 1 au début de chaque année lorsque $(x)$ survit, à partir de l'âge $x + n$. La variable aléatoire est 
	$$Y = \begin{cases}
	0,& 0\leq K < n\\
	\ax**[n|]{\angl{K + 1 - n}}, & K \geq n
	\end{cases}$$
	L'espérance de la variable aléatoire est 
	$$\ax[n|]{x} = \sum_{k = n}^{\infty} v^k \px[k]{x} = \Ex[n]{x} \ax**{x + n} = \ax**{x} - \ax**{\endowxn}.$$
\end{definition}

\begin{definition}{Rente viagère garantie $n$ années}{}
	Une rente viagère garantie $n$ années paie une rente certaine pour $n$ ans (peu importe si l'assuré survie ou non) et continue les paiements jusqu'au décès. La variable aléatoire de la valeur actualisée des paiements est 
	$$Y = \begin{cases}
	\ax**{\angln}, & 0\leq K < n\\
	\ax**{\angl{K + 1}}, & K \geq n
	\end{cases}$$
	$$\ax*{\joint\endowxn} = \ax**{\angln} \qx[n]{x} + \sum_{k = n}^{\infty} \ax**{\angl{k + 1}} \px[k]{x}q_{x+k} = \ax**{\angln} + \sum_{k = n}^{\infty} v^k \px[k]{x}$$
\end{definition}

\subsection{Rentes payables en fin d'année}

Des versions fin de période sont disponibles pour les rentes. La variable aléatoire qui représente la valeur actualisée des paiements de ces rentes de type
$$ Y = \ax{\angl{T}}$$
et l'espérance de cette variable aléatoire est 
$$a_x = \sum_{k = 0}^{\infty} \px[k]{x} q_{x + k} \ax{\angl{k}} = \sum_{k = 1}^{\infty} v^k \px[k]{x}.$$
Le lien entre cette variable aléatoire et une assurance est 
$$a_x = \frac{1 - (1 + i)A_x}{i}.$$

\section{Rentes payables $m$ fois par année}

\begin{definition}{}{}
	Une rente qui paie $m$ fois par année le montant $\frac{1}{m}$ chaque période où l'assuré survit est noté $\ax**{x}[(m)]$. La variable aléatoire de la valeur actualisée est
	$$Y = \sum_{j = 0}^{mK + J}\frac{1}{m}v^{j/m} = \ax**{\angl{K + (J + 1)/m}}[(m)] = \frac{1 - v^{K + (J + 1)/m}}{d^{(m)}},$$
	où $K$ est le nombre d'années complètes et $J + 1$ est le nombre de paiements faits dans l'année de décès. L'espérance de cette variable aléatoire est 
	$$\ax**{x}[(m)] = \frac{1}{m}\sum_{h = 0}^{\infty} v^{h/m}\px[h/m]{x} = \frac{1 - A_x^{(m)}}{d^{(m)}}.$$
\end{definition}

\subsection{Approximations}

On peut utiliser des approximations sur la mortalité pour obtenir des approximations des rentes payables $m$ fois par année à partir des rentes payables une fois par année. 

\begin{proposition}{}{dud-mfois}
	Sous DUD, on a 
	$$\ax**{x}[(m)] = \alpha(m)\times \ax**{x} - \beta(m),$$
	où 
	$$\alpha(m) = \sx{\angl{1}}[(m)]\ax**{\angl{1}}[(m)] = \frac{i}{i ^{(m)}} \frac{d}{d ^{(m)}}$$
	et
	$$\beta(m) = \frac{\sx{\angl{1}}[(m)] - 1}{d^{(m)}} = \frac{i - i^{(m)}}{i^{(m)}d^{(m)}}$$
\end{proposition}

\begin{proposition}{}{lineaire-mfois}
	Sous l'hypothèse que $v^{k + j/m}\px[k + j/m]{x}$ est linéaire, on a 
	$$\alpha(m) = 1$$
	et
	$$\beta(m) = \frac{m-1}{2m}$$
\end{proposition}

Connaître la peuve des propositions \ref{prop:dud-mfois} et \ref{prop:lineaire-mfois}.

