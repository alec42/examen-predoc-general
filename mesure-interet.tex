\chapter{La mesure de l'intérêt}

\section{Les fonctions d'accumulation et de montant}

\begin{itemize}
	\item Le montant initial investi est appelé le \textit{capital}
	\item Le montant obtenu après une période de temps est appelé la \textit{valeur accumulée}
	\item La différence entre le montant accumulé et le principal est le \textit{montant d'intérêt} ou \textit{intérêt}
	\item $t$ est le temps mesuré depuis le début la date d'investissement
	\item L'unité de temps est la \textit{période de mesure} ou \textit{période}
\end{itemize}

\begin{definition}{La fonction d'accumulation}{}
	La fonction d'accumulation $a(t)$ correspond à la valeur accumulée au temps $t\geq 0$ d'un investissement initial de $1$. 
\end{definition}

La fonction d'accumulation possède les propriété suivantes : 
\begin{enumerate}
	\item $a(0) = 1$
	\item $a(t)$ est généralement croissante
	\item Si l'intérêt est accumulé en continu, $a(t)$ est continue
\end{enumerate}

\begin{definition}{Fonction de montant}{}
	La fonction de montant $A(t)$ correspond à la valeur accumulée au temps $t\geq 0$ d'un investissement initial de $k>0$. On a 
	$$A(t) = k \times a(t)$$
	et
	$$A(0) = k.$$
\end{definition}

\begin{definition}{Montant d'intérêt}{}
	Le montant d'intérêt gagné pendant la $n^{\text{ième}}$ période depuis la date d'investissement est $$I_n = A(n) - A(n-1) \quad \text{pour} \quad n\geq 1.$$
\end{definition}

\section{Le taux d'intérêt effectif}

\begin{definition}{}{}
	Le taux d'intérêt effectif $i$ est le montant d'argent qu'une unité investie au début de la période va gagner pendant la période, où l'intérêt est obtenu à la fin de la période, c.-à-d. $i = a(1) - a(0)$ ou $a(1) = 1 + i$.
	\tcblower
	Le taux d'intérêt $i$ est le ratio du montant d'intérêt gagné pendant la période et du montant de principal investi au début de la période. 
\end{definition}

Le mot effectif est utilisé lorsque l'intérêt est payé une fois par période. 

Les taux d'intérêt peuvent être calculés sur n'importe quelle période d'investissement. Soit $i_n$, le taux d'intérêt effectif pour la $n^\text{ième}$ période depuis la date d'investissement. Alors, on a 
$$i_n = \frac{A(n) - A(n-1)}{A(n-1)} = \frac{I_n}{A(n-1)}$$
pour $n = 1, 2, \dots$

\section{Intérêt simple}

La fonction d'accumulation est linéaire
$$a(t) = a + it \quad \text{pour} \quad t\geq 0.$$ Accumuler de l'intérêt avec ce patron correspond à l'intérêt simple. On a 
$$i_n = \frac{i}{1 + i(n-1)},$$
donc un intérêt simple constant implique un taux d'intérêt effectif décroissant. 

\section{Intérêt composé}

L'intérêt composé assume que le montant accumulé est automatiquement ré-investi. 

$$a(t) = (1 + i)^t \quad \text{pour} \quad t\geq 0.$$

On a $i_n = i$, qui est indépendant de $n$. Alors, un taux d'intérêt composé correspond au taux d'intérêt effectif. 

\section{Valeur actualisée}

Le terme $1 + i$ est le facteur d'accumulation, car il accumule la valeur d'un investissement au début de la période à la fin de la période. On doit parfois déterminer le montant à investir pour obtenir 1 à la fin de la période. Pour ce faire, on définit un nouveau symbole $$v = \frac{1}{1 + i}$$ parfois appelé le facteur d'escompte. 

\begin{itemize}
	\item La fonction d'escompte est 
	$$a^{-1}(t) = \frac{1}{a(t)}.$$
	\item Pour l'intérêt simple, on a 
	$$a^{-1}(t) = \frac{1}{1 + it}.$$
	\item Pour l'intérêt composé, on a 
	$$a^{-1}(t) = \frac{1}{(1 + i)^t} = v^t.$$
\end{itemize}

\section{Taux effectif d'escompte}

\begin{definition}{Le taux effectif d'escompte}{}
	Le taux effectif d'escompte $d$ est le ratio du montant d'intérêt gagné pendant la période et du montant investi à la fin de la période.
\end{definition}

La principale différence entre le taux d'effectif d'intérêt et le taux effectif d'escompte est 
\begin{itemize}
	\item Intérêt : payé à la fin de la période divisé par la balance au début de la période.
	\item Escompte : payé au début de la période divisé sur la balance à la fin de la période. 
\end{itemize}

$$d_n = \frac{A(n) - A(n-1)}{A(n)} = \frac{I_n}{A(n)}, \quad \text{pour un entier} \quad n \geq 1.$$

\begin{definition}{Équivalence}{}
	Deux taux d'intérêt ou d'escompte sont dit équivalant si un montant de principal investi pour la même durée à chaque taux d'intérêt produisent la même valeur accumulée. 
\end{definition}

On utilise le concept d'équivalence pour établir des liens entre les taux d'intérêt. 

Si une personne emprunte 1 au taux d'escompte effectif $d$, le principal initial est $1-d$ et le montant d'intérêt est $d$. Alors, 
$$i = \frac{d}{1-d} \Rightarrow d = \frac{i}{1 + i} = iv.$$
Autres relations utiles : 
$$d = \frac{i}{1 + i} = \frac{1 + i}{1 + i} - \frac{1}{1 + i} = 1 - v;$$
$$d = iv = i(1-d) = i - id \Rightarrow i - d = id.$$

\section{Taux d'intérêt et d'escompte nominaux}

\begin{itemize}
	\item On considère les situations où l'intérêt est payé plus fréquemment qu'une fois par période. Ces taux sont appelés nominaux.
	\item Le symbole pour un taux nominal d'intérêt payé $m$ fois par période est $i^{(m)}$, où $m$ est un entier positif. 
	\item Par un taux nominal d'intérêt $i^{(m)}$, on veut dire que l'intérêt est $i^{(m)}/m$ pour chaque $\frac{1}{m}$ période. 
	\item De la définition d'équivalence, on a 
	$$1 + i = \left(1 + \frac{i^{(m)}}{m}\right)^m \Rightarrow i^{(m)} = m\left[(1 + i)^{\frac{1}{m}} - 1\right]$$
	\item Le symbole pour un taux nominal d'escompte payé $m$ fois par période est $d^{(m)}$, où $m$ est un entier positif. 
	\item De la définition d'équivalence, on a 
	$$1 - d = \left(1 - \frac{d^{(m)}}{m}\right)^m \Rightarrow d^{(m)} = m\left[1 - v^{\frac{1}{m}}\right]$$
	\item Une autre relation est 
	$$\frac{i^{(m)}}{m} - \frac{d^{(m)}}{m} = \frac{i^{(m)}}{m} \frac{d^{(m)}}{m}.$$
\end{itemize}








