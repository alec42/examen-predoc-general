\chapter{Estimation for complete data}

Lorsque les observations sont collectées d'une distribution de probabilité, la situation idéale est d'avoir les réalisations de chaque observation. Ce cas est référé aux données complètes individuelles. 

Les données exactes peuvent être impossibles à avoir car
\begin{enumerate}
	\item les données peuvent être groupées;
	\item les données peuvent être censurées ou tronquées. 
\end{enumerate}

On considère premièrement l'estimation pour les données complètes et individuelles. 

\begin{definition}{}{}
	La fonction de distribution empirique est 
	$$F_n(x) = \frac{\text{Nombre d'observations }\leq x}{n},$$
	où $n$ est le nombre total d'observations. 
\end{definition}

\begin{definition}{}{}
	La fonction de hasard cumulative est définie selon 
	$$H(x) = -\ln S(x).$$
	\tcblower
	Source de la fonction : 
	$$H'(x) = -\frac{S'(x)}{S(x)} = \frac{f(x)}{S(x)} = h(x) \Rightarrow H(x) = \int_{-\infty}^{x} h(y) dy.$$
\end{definition}

Autre notation : 

\begin{itemize}
	\item Soit $n$, la taille échantillonnalle
	\item Soit $y_1<y_2<\dots<y_k$ être les $k$ valeurs uniques
	\item Soit $s_j$, le nombre de gois dont $y_j$ apparraît dans l'échantillon. 
	\item Soit le nombre d'observations plus grande qu'une valeur. L'ensemble des risques $r_j$ est définie par $r_j = \sum_{i = j}^{k}s_j$, le nombre d'observations plus grand ou égal à $y_j$.
\end{itemize}
En utilisant cette notation, on obtient
$$F_n(x) = \begin{cases}
0,&x<y_1\\
1-\frac{r_j}{n}, & y_{j-1}\leq x < y_j,\quad j = 2, 3, \dots, k\\
1,& x\geq y_k
\end{cases}$$

\begin{definition}{L'estimateur Nelson-\AA{}alen}{}
	L'estimateur Nelson-\AA{}alen de la fonction de hasard cumulative est 
	$$\hat{H}(x) = \begin{cases}
	0,&x<y_1\\
	\sum_{i = 1}^{j-1} \frac{s_i}{r_i}, & y_{j-1}\leq x < y_j,\quad j = 2, 3, \dots, k\\
	\sum_{i = 1}^{k} \frac{s_i}{r_i},& x\geq y_k
	\end{cases}$$
\end{definition}