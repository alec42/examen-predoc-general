\part{Théorie de la crédibilité}

\chapter{Crédibilité Bayesienne}


\chapter{Le modèle de Buhlmann}

\section{La prime de crédibilité}

Objectif : minimiser l'erreur quadratique moyenne de la prime de crédibilité avec un estimateur linéaire de la forme $$\mu(\theta) = \alpha_0 + \sum_{j = 1}^{n}\alpha_j X_j.$$

On souhaite minimiser $$Q = E\left[ \left(\mu_{n+1}(\theta) - \alpha_0 - \sum_{j = 1}^{n}\alpha_j X_j\right)^2\right],$$
par rapport à $\alpha_0, \alpha_1, \dots, \alpha_n$. L'espérance est prise par sur la distribution conjointe de $\Theta, \underline{X}$. 

En dérivant et en jouant avec les espérances, on a 
$$E[X_{n+1}] = \tilde{\alpha}_0 + \sum_{j = 1}^{n}\tilde{\alpha}_j X_j$$
et
$$Cov(X_i, X_{n+1}) = \sum_{j = 1}^{n}\tilde{\alpha}_j Cov(X_i, X_j).$$

\section{Le modèle de Buhlmann}

Notation : 

\begin{multicols}{2}
	\begin{itemize}
		\item $\displaystyle \mu(\theta) = E[X_j \vert \Theta = \theta]$
		\item $\displaystyle v(\theta) = Var(X_j \vert \Theta = \theta)$
		\columnbreak
		\item $\displaystyle \mu = E[\mu(\Theta)]$
		\item $\displaystyle v = E[v(\Theta)]$
		\item $\displaystyle a = Var(\mu(\Theta))$
	\end{itemize}
\end{multicols}

Résultats intermédiaires (obtenus en conditionnant) :

\begin{itemize}
	\item $\displaystyle E[X_j] = \mu$
	\item $\displaystyle Var(X_j) = v + a$
	\item $\displaystyle Cov(X_j, X_i) = a$.
\end{itemize}

La prime de crédibilité devient

$$\tilde{\alpha}_0 + \sum_{j = 1}^{n}\tilde{\alpha}_jX_j = Z\overline{X} + (1-Z)\mu,$$
avec
$$Z = \frac{n}{n + k}$$
et
$$k = \frac{v}{a} = \frac{E[Var(X_j \vert \Theta)]}{Var(E[X_j \vert \Theta])}.$$

\chapter{Le modèle de Buhlmann-Straub}


















