\chapter{Rappel des notions de mathématiques statistiques}

\section{Estimation ponctuelle}

Dans cette section, on s'intéresse à produire une seule valeur pour déterminer la valeur d'une quantité pour une population inconnue. 

\subsection{Mesures de qualité}

\begin{definition}{Biais}{}
	Un estimateur $\hat{\theta}$ est sans biais si $E[\hat{\theta}\vert \theta] = \theta, ~\forall ~\theta$. Le biais est $$\text{bias}_{\hat{\theta}}(\theta) = E[\hat{\theta}\vert \theta] - \theta.$$
\end{definition}

\begin{definition}{Biais asymptotique}{}
	Soit $\hat{\theta}_n$, un estimateur de $\theta$ basé sur un échantillon de taille $n$. L'estimateur est asymptotiquement sans biais si $$\lim\limits_{n\to \infty }E[\hat{\theta}_n\vert \theta] = \theta, ~\forall \theta.$$
\end{definition}

\begin{definition}{Convergence}{}
	Un estimateur est convergent si, pour tout $\delta > 0$ et pout tout $\theta$, 
	$$\lim\limits_{n\to \infty} \Pr\left(|\hat{\theta}_n - \theta| > \delta\right) = 0.$$
	\tcblower
	\begin{itemize}
		\item Une condition suffisante (mais pas nécessaire) est que l'estimateur soit asymptotiquement sans biais et $Var\left(\hat{\theta}_n\right) \to 0$.
		\item Critique : le choix de $\delta$ est parfois arbitraire. 
	\end{itemize}
\end{definition}

\begin{definition}{Erreur quadratique moyenne}{}
	L'erreur quadratique moyen d'un estimateur est 
	$$MSE_{\hat{\theta}}(\theta) = E\left[\left(\hat{\theta} - \theta \right)^2 \vert \theta \right].$$
	\tcblower
	On peut décomposer cet estimateur comme
	$$MSE_{\hat{\theta}}(\theta) = E\left[ \left.\left(\hat{\theta} - E[\hat{\theta}\vert \theta] + E[\hat{\theta}\vert \theta] - \theta\right)^2 \right\vert \theta  \right] = Var\left(\hat{\theta}\vert \theta\right) + \left[\text{bias}_{\hat{\theta}}(\theta)\right]^2$$
\end{definition}

\begin{definition}{Estimateur sans biais à variance minimale}{}
	Un estimateur $\hat{\theta}$ est appelé estimateur sans biais à variance minimale si il est sans biais et pour la vraie valeur de $\theta$ il n'existe pas d'estimateur avec une plus faible variance. 
\end{definition}

\section{Intervalles de confiance}

Dans cette section, on s'intéresse à produire un intervalle de valeurs possibles d'une quantité pour une population inconnue. 

\begin{definition}{}{}
	Un intervalle de confiance à $100(1-\alpha)\%$ pour un paramètre $\theta$ est une paire de valeurs aléatoires, $L$ et $U$, calculé d'un échantillon aléatoire tel que $$\Pr(L\leq \theta \leq U) \geq 1 - \alpha, ~\forall \theta.$$
\end{definition}

Si la population est normale avec moyenne et variance inconnue, on a 
$$L = \overline{X} - t_{\alpha / 2, n-1} \frac{s}{\sqrt{n}}, \quad U = \overline{X} + t_{\alpha / 2, n-1} \frac{s}{\sqrt{n}}$$
et
$$s = \sqrt{\frac{\sum_{j = 1}^{n}\left(X_j - \overline{X}\right)^2}{n-1}}.$$

Sinon, l'approximation normale peut être utilisée. Si $\hat{\theta}$ est approximativement normal, on a (approximtivement)
$$1 - \alpha = \Pr \left( -z_{\alpha / 2} \leq \frac{\hat{\theta} - \theta}{\sqrt{v(\theta)}} \leq z_{\alpha / 2}\right).$$
On remarque que $\theta$ apparaît à deux endroits. Alors, si isoler $\theta$ est trop compliqué, on peut utiliser l'approximation $v(\theta) \approx v(\hat{\theta})$ et l'intervalle de confiance devient
$$1 - \theta = \Pr\left(\theta - z_{\alpha / 2} \sqrt{v(\hat{\theta})} \leq \theta \leq \theta + z_{\alpha / 2} \sqrt{v(\hat{\theta})}\right).$$

\section{Tests d'hypothèse}

Un test d'hypothèse est composé de deux hypothèses :
\begin{itemize}
	\item l'hypothèse nulle (notée $H_0$), et
	\item l'hypothèse alternative (notée $H_1$).
\end{itemize}

Les deux hypothèses ne sont pas symmétriques : les alterner peut changer les résultats. 

\begin{itemize}
	\item Déterminer l'hypothèse est faite avec une statistique de test. C'est une fonction des observations et est une variable aléatoire. 
	\item La spécification d'un test est complétée en construisant une région de rejet. 
	\item Les bornes de la région (autre que $\pm \infty$) sont appelées les valeurs critiques. 
\end{itemize}

\subsection{Types d'erreur.}

Lorsqu'on construit un test, on peut avoir deux types d'erreurs. Le premier se produit si on rejette l'hypothèse nulle mais que l'hypothèse nulle était vraie. On appelle cet erreur \textit{type I}. 

\begin{definition}{Niveau d'importance}{}
	Le niveau d'importance (exceptionnel) d'un test d'hypothèse est la probabilité de faire un erreur de type I sachant que l'hypothèse nulle est vraie. On note ce niveau $\alpha$. Le niveau d'importance est habituellement déterminé d'avance, entre 0.01 et 0.1. 
\end{definition}

Le deuxième type d'erreur est de ne pas rejetter l'hypothèse nulle alors que l'hypothèse alternative est vraie. La puissance du test est 1 - la probabilité de faire un erreur de type II. 

\begin{definition}{}{}
	Un test d'hypothèse est uniformément plus puissant s'il n'existe aucun autre test qui a le même ou plus bas niveau d'importance et, pour une valeur particulière dans l'hypothèse alternative, a une plus petite probabilité de faire un erreur de type II. 
\end{definition}
Généralement, diminuer un type d'erreur cause l'autre à augmenter. 

\begin{definition}{}{}
	Pour un test d'hypothèse, la valeur-$p$ est la probabilité que la statistique de test prend une valeur moins d'accord avec l'hypothèse nulle que la valeur obtenue dans l'échantillon. Les tests effectués à un niveau de confiance supérieur à la valeur-$p$ mènent à rejeter l'hypothèse nulle, et les test effectués à un niveau de confiance infétieur à la valeur-$p$ mènent à l'échec de rejeter l'hypothèse nulle. 
\end{definition}
