\chapter{Mutualisation des risques}

\section{Aggrégation des risques}

On s'intéresse au comportement de la v.a. $S = \sum_{i = 1}^{n}X_i$ représentant les coûts  d'un portefeuille de $n$ risques. On a 
$$E[S] = E\left[\sum_{i = 1}^{n} X_i\right] = \sum_{i = 1}^{n}E[X_i]$$
et
$$Var(S) = Var\left(\sum_{i = 1}^{n}X_i\right) = \sum_{i = 1}^{n} Var(X_i) + \underset{i \neq j}{\sum_{i = 1}^{n}\sum_{j = 1}^{n}}Cov(X_i, X_j).$$
La fgm est donnée par 
$$M_S(t) = E\left[e^{t\sum_{i = 1}^{n}X_i}\right] \stackrel{ind}{=} \prod_{i = 1}^{n} E\left[e^{tX_i}\right] = \prod_{i = 1}^{n}M_{X_i}(t).$$
Si les v.a. $X_i$ sont iid, on a 
$$M_S(t) = M_{X_i}(t)^n.$$ 

\begin{proposition}{Loi Poisson composée et aggrégation}{}
	Soient les v.a. indépendantes $X_1, \dots, X_n$ où
	$$X_i \sim PComp(\lambda_i, F_{B_i}), i = 1, 2, \dots, n.$$
	Alors, 
	$$S = \sum_{i = 1}^{n} X_i \sim PComp(\lambda_S, F_{C}),$$
	où $\lambda_S = \sum_{i = 1}^{n}\lambda_i$ et 
	$$F_{C}(s) = \frac{\lambda_1}{\lambda_S}F_{B_1}(x) + \frac{\lambda_2}{\lambda_S}F_{B_2}(x) + \dots + \frac{\lambda_n}{\lambda_S}F_{B_n}(x)$$
	\tcblower
	Preuve : fgm et mettre $\lambda_S$ en évidence. 
\end{proposition}

\begin{proposition}{Loi binomiale négative composée et aggrégation}{}
	Soient les v.a. indépendantes $X_1, \dots, X_n$ où
	$$X_i \sim BNComp(r_i, q, F_{B_i}), i = 1, 2, \dots, n.$$
	Alors, 
	$$S = \sum_{i = 1}^{n} X_i \sim BNComp(r_S, q, F_{B}),$$
	où $r_S = \sum_{i = 1}^{n}r_i$.
\end{proposition}

\section{Méthodes d'approximation basée sur les moments}

\subsection{Approximation basée sur la distribution normale}

L'approximation basée sur la distribution normale est fondée sur le théorème central limite. On approxime la v.a. $S$ par la v.a. $T$, en supposant $E[S]$ et $Var(S)$ connus. On a alors
$$T \sim Norm(E[S], Var(S)).$$
On obtient alors
\begin{align*}
	F_S(x)         & \approxeq F_T(x) = \Phi\left(\frac{x - E[S]}{\sqrt{Var(S)}}\right) \\
	VaR_\kappa(S)  & \approxeq VaR_\kappa(T) = E[S] + \sqrt{Var(S)} VaR_\kappa(Z)       \\
	TVaR_\kappa(S) & \approxeq TVaR_\kappa(T) = E[S] + \sqrt{Var(S)} TVaR_\kappa(Z).
\end{align*}

\subsection{Approximation basée sur la loi gamma translatée}

L'approximation basée sur la loi gamma est utile pour approximer la distribution de la v.a. $S$ pour la queue de droite. On approxime la v.a. $S$ par $T + x_0$, où $T + x_0$ obéit à une loi gamma translatée. 

À partir des trois premiers moments, on a 
\begin{align*}
E[S] &= \frac{\alpha}{\beta} + x_0\\
Var(S) &= \frac{\alpha}{\beta^2}\\
\gamma(S) = \frac{2}{\alpha^{0.5}}.
\end{align*}
Avec la méthode des moments, on en déduit que
\begin{align*}
\hat{\alpha} &= \frac{4}{\gamma(S)^2}\\
\hat{\beta} &= \frac{2}{\gamma(S) \sqrt{Var(S)}}\\
\hat{x_0} &= E[S] - \frac{2\sqrt{Var(S)}}{\gamma(S)}
\end{align*}
On peut approximer les mesures de risque suivantes pour $\kappa$ élevés : 
\begin{align*}
	VaR_\kappa(S)  & \approxeq VaR_\kappa(T) + x_0  \\
	TVaR_\kappa(S) & \approxeq TVaR_\kappa(T) + x_0
\end{align*}

\subsection{Autres approximations}

\begin{itemize}
	\item lognormale : $\sigma_T = \sqrt{\ln \frac{E[S^2]}{E[S]^2}}$ et $\mu = \ln(E[S]) - \frac{1}{2}\sigma_T^2$
	\item F-généralisée.
\end{itemize}

\subsection{Approximation Poisson du modèle individuel}

\begin{itemize}
	\item Rappel : $S = X_1 + X_2 + \dots + X_n$, où $X_j = I_j \times b_j$. On a $E[S] = \sum_{j = 1}^{n}b_j q_j$ et $Var(S) = \sum_{j = 1}^{n}b_j^2 q_j(1-q_j)$.
	\item Si $q_j$ est petit, on peut approximer $I_j$ par une loi de Poisson en gardant la même moyenne $\lambda_j = q_j$. Une autre approximation de la loi de Poisson est de garder la même probabilité de non-paiement $e^{-\lambda_j} = 1 - q_j \Rightarrow \lambda_j = -\ln(1-q_j)$. 
	\item Si $\Pr(B = b_j) = 1$ (comme en assurance vie), alors 
	$$f_S(x) = \Pr(X = x) = \frac{1}{\lambda_S}\sum_{\{j: b_j = x\}} \lambda_j$$
	
\end{itemize}
\section{Mutualisation et activités d'assurance}

\begin{definition}{Coût moyen par contrat}{}
	Le coût moyen par contrat est 
	$$W_n = \frac{S_n}{n}.$$
	On a 
	$$E[W_n] = E\left[\frac{S_n}{n}\right] = \frac{1}{n}E[S_n] = E[X]$$
	La variance est donnée par 
	$$Var(W_n) = Var\left(\frac{S_n}{n}\right) = \frac{1}{n^2}Var(S_n) \stackrel{\text{ind}}{=} \frac{1}{n}Var(X).$$
	On remarque que $Var(W_n)$ converge vers 0 lorsque $n$ tend vers l'infini. 
\end{definition}

Si les v.a. ne sont pas indépendantes, on a 
$$Var(W_n) = \frac{1}{n^2}[nVar(X) + n(n-1) Cov(X_1, X_2)] = \frac{Var(X)}{n} + \frac{n-1}{n}Cov(X_1, X_2).$$
On remarque que $Var(W_n)$ converge vers $Cov(X_1, X_2)$ lorsque $n$ tend vers l'infini. 

\begin{proposition}{}{}
	Soient les risques $X_1, X_2, \dots, X_n$ et la prime $\pi_i = (1 + \eta_i)E[X_i]$. On définit le coefficient $\rho_B$ comme étant la solution strictement positive (si elle existe) de $E[e^{tS}] = e^{t\pi_S}$, où 
	$$\psi_n(u) \leq e^{-\rho_B u}, \quad u\geq 0.$$
\end{proposition}

\begin{proposition}{}{}
	Soient les risques iid $X_1, X_2, \dots, X_n$ et la prime $\pi_i = (1 + \eta_i)E[X_i] = \pi_i$. Alors, on a 
	\begin{enumerate}
		\item si $\pi > E[X], \lim\limits_{n \to \infty}\phi_n(u) = 0, u\geq 0$
		\item si $\pi < E[X], \lim\limits_{n \to \infty}\phi_n(u) = 1$
	\end{enumerate}
\end{proposition}












