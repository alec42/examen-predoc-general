\chapter{Solutions de problèmes d'intérêt}

Voir \cite{kellison2006theory}, chapitre 2.

\section{Le problème de base}

Les problèmes d'intérêts sont composés de quatre éléments : 
\begin{enumerate}
	\item Le principal investi initialement
	\item La longueur de la période d'investissement
	\item Le taux d'intérêt
	\item La valeur accumulée du principal à la fin de la période d'investissement.
\end{enumerate}
Connaissant trois éléments, le quatrième peut être résoud.

\section{Équations de valeur}

\begin{itemize}
	\item Principe fondamental : la valeur temporelle de l'argent. 
	\item Deux valeurs monétaires à différents temps ne peuvent pas être comparés. 
	\item On doit accumuler ou escompter les valeurs à une \textit{date de comparaison}
	\item L'équation qui compare deux valeurs monétaires à la même date de comparaison est l'équation de valeur. 
\end{itemize}