\chapter{Assurance vie}

\section{Assurance continue}

La variable aléatoire qui représente l'assurance vie est notée $Z = b_T v_T$, où $b_t$ est une fonction de prestation et $v_t$ est une fonction d'escompte. La variable aléatoire $T$ représente la durée de vie de l'assuré, comme présenté dans le chapitre sur les distributions de survie. 

\begin{definition}{Assurance vie temporaire $\exercicemath{n}$-années}{}
	Cette assurance paie 1 si l'assuré décède pendant les $n$ premières années et 0 sinon. La variable aléatoire est
	$$Z = \begin{cases}
	v^T,& T\leq n\\
	0,& T>n
	\end{cases}$$
	
	L'espérance de cette variable aléatoire est  
	$$\Ax*[][]{\termxn} = E[Z] = \int_{0}^{n}v^t \px[t]{x}\mu_x(t) dt.$$
	
	Le $j^{\text{ème}}$ moment est
	
	$$\Ax*[][j]{\termxn} = E\left[Z^j\right] = \int_{0}^{n}\left(v^t\right)^j \px[t]{x}\mu_x(t) dt,$$
	donc est égal au premier moment mais où la force d'intérêt est multipliée par $j$. De plus, on a 
	$$F_{Z}(x) = \begin{cases}
	\overline{F}_{T_x}(n),& x = 0\\
	\overline{F}_{T_x}(n),& 0<x<bv^n\\
	\overline{F}_{T_x}\left(-\frac{1}{\delta}\ln \left(\frac{x}{b}\right)\right),& bv^n<x<b\\
	1,& x>b
	\end{cases}$$
	On déduit que 
	$$VaR_\kappa(Z) = \begin{cases}
	0,& 0<\kappa < \overline{F}_{T_x}(n)\\
	bv^{VaR_{1-\kappa}(T_x)},& \overline{F}_{T_x}(n)<\kappa<1
	\end{cases}$$
	et
	$$TVaR_{\kappa}(Z) = \begin{cases}
	\frac{1}{1 - \kappa}b\Ax*{\termxn},& 0<\kappa < \px[n]{x}\\
	\frac{1}{1 - \kappa}b\Ax*{\nthtop{1}{x}:\angl{VaR_{1-\kappa}(T_x)}},& \px[n]{x}<\kappa<1\\
	\end{cases}$$
\end{definition}

\begin{definition}{Assurance vie complète}{}
	La variable aléatoire pour une assurance vie complète est $$Z=v^T, t\geq 0.$$ C'est un cas particulier de l'assurance vie temporaire avec $n\to \infty$. On a 
	$$\Ax*[][j]{x} =  \int_{0}^{\infty}v^{jt} \px[t]{x}\mu_x(t) dt.$$
\end{definition}

\begin{definition}{Capital différé}{}
	Un capital différé verse une prestation après $n$ années si l'assuré a survécu les $n$ années. La v.a. est
	$$Z = \begin{cases}
	0&, T\neq n\\
	v^n&, T > n
	\end{cases}$$	
	La notation est 
	$$\Ax*{\pureendowxn} = \Ex[n]{x}= v^n \px[n]{x}.$$
\end{definition}

\begin{definition}{Assurance capital différé (assurance mixte)}{}
	Une assurance capital différé verse une prestation au décès ou après $n$ années. La v.a. est
	$$Z = \begin{cases}
	0&, T\neq n\\
	v^n&, T > n
	\end{cases}$$	
	La notation est 
	$$\Ax*{\pureendowxn} = \Ex[n]{x}= v^n \px[n]{x}.$$
\end{definition}

\section{Assurance discrète}

Une assurance discrète est une assurance payable à la fin de l'année de décès. 
\begin{definition}{}{}
	La variable aléatoire pour une assurance temporaire $n$ années est 
	$$Z = \begin{cases}
	v^{K + 1}, & K = 0, 1, 2, \dots, n-1\\
	0, & \text{sinon}
	\end{cases}$$
	L'espérance de la valeur actualisée de cette variable aléatoire est 
	$$\Ax{\termxn} = \sum_{k = 0}^{n-1}v^{k + 1}\px[k]{x}\qx[]{x+k}.$$
	Avec la règle des moments, on a 
	$$Var\left(Z\right) = \Ax[][2]{\termxn} - \left(\Ax{\termxn}\right),$$
	où $$\Ax[][2]{\termxn} = \sum_{k = 0}^{n-1}v^{2k + 2}\px[k]{x}\qx[]{x+k}$$
\end{definition}

Des relations de récursion existent aussi pour une assurance discrète : 
$$\Ax{\termxn} = vq_x + vp_x \Ax{\nthtop{1}{x+1}:\angl{n-1}}$$
Intuition : une assurance temporaire $n$ années est égale à une assurance temporaire un an plus une assurance temporaire $n-1$ ans à un assuré d'âge $x+1$ dans un an.

\begin{definition}{}{}
	La variable aléatoire pour une assurance capital différé payable à la fin de l'année est 
	$$Z = \begin{cases}
	v^{K + 1}, & K = 0, 1, 2, \dots, n-1\\
	v^n, & K = n, n+1, \dots
	\end{cases}$$
	La valeur actuarielle actualisée est 
	$$\Ax{\endowxn} = \sum_{k = 0}^{n-1}v^{k+1}\px[k]{x}\qx[]{x+k} + v^n \px[n]{x}$$
\end{definition}

\section{Relation entre l'assurance payable au moment du décès et la fin de l'année}

Avec l'hypothèse DUD, on a 
$$\Ax*{x} = \frac{i}{\delta}A_x.$$
De manière similaire, on a 
$$\Ax{x}[(m)] = \frac{i}{i^{(m)}}A_x.$$















